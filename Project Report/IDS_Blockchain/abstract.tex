\vspace{2in}
\begin{abstract}

In an attempt to cope with the increased number of cyberattacks,with the purpose of identifying cyber threats and possible incidents, intrusion detection systems (IDSs) are widely deployed in various computer networks.IDS is used to prevent the attacker and keep the data  save as possible.  In order to enhance the detection capability of a single IDS, collaborative intrusion detection networks  (or collaborative IDSs) have been developed, which allow IDS nodes to exchange data with each other. Distributed ledger technologies, e.g. various implementations of blockchain protocols, are a good fit to the problem of enhancing trust in collaborative environments.In recent years, blockchain technology has shown its adaptability in many fields, such as supply chain management, international payment, interbanking, and so on. As blockchain can protect the integrity of data storage and ensure process transparency, it has a potential to be applied to intrusion detection domain. This project is all about the intersection of IDSs and blockchains  as in how blockchain technology is used in IDS.

\end{abstract} 
